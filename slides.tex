\documentclass[aspectratio=1610,t]{beamer}

% Colors
\usepackage{color}
\definecolor{mainorange}{HTML}{EC811B}
\definecolor{lightgrey}{HTML}{888888}

% Syntax highlighting
\usepackage{minted}
\usepackage{alltt}
\newcommand\hi[1]{{\color{mainorange} \textbf{#1}}}

% Theme
\usetheme[%
	subsectionpage=progressbar,
	numbering=fraction,
	progressbar=foot,
]{metropolis}

% Customization
\setbeamertemplate{section in toc}[sections numbered]
\setbeamerfont{title}{size=\fontsize{30}{30}}
\setbeamerfont{block title}{size=\large}
\newcommand\sep{\textcolor{lightgrey}{\rule{\linewidth}{0.05mm}}}

% Meta
\title{Thread pools and iterators}
\date{\today}
\author{Stefan Schindler (@dns2utf8)}
\institute{Rust Zürichsee, Schweiz CH}

\begin{document}

\pgfdeclareimage[width=\paperwidth]{bg}{background-light.pdf}
\pgfdeclareimage[width=\paperwidth]{bgdark}{background-dark.pdf}

\usebackgroundtemplate{\pgfuseimage{bgdark}}
\maketitle

% ----------------------------------------------------------------- %

\begin{frame}[plain,noframenumbering]
	\frametitle{Inhalt}
	\setcounter{tocdepth}{1}
	\tableofcontents
\end{frame}

% ----------------------------------------------------------------- %

\usebackgroundtemplate{\pgfuseimage{bg}}

{
\usebackgroundtemplate{\pgfuseimage{bgdark}}
\section{Über}
}

%\begin{frame}[fragile]{Timetable}
%  \begin{itemize}
%    \item now => Talk
%    \item 20:00 => Questions
%    \item 20:10 => Happy hacking
%    \item 21:00 => Closing
%    \item tomorrow => ???
%    \item the day after => Parallelize the World!
%  \end{itemize}
%\end{frame}
% timetable


\begin{frame}[fragile]{About:me}
Hallo mein Name ist Stefan und I arbeite an und mit Computern.

Ich organisier
\begin{itemize}
  \item RustFest.eu Paris: 26. \& 27. May mit "impl days" am 28. \& 29. May
  \item Meetups in und um Zürich
  \item Illuminox.ch (in den Schweizer Alpen Juli 2018)
\end{itemize}

Ein paar von meinen Nebenprojekten
\begin{itemize}
  \item rust threadpool
  \item Son of Grid Engine (SGE) interface
  \item run your own infrastructure - DNS, VPN, Web, ...
\end{itemize}
\end{frame}



\begin{frame}[fragile]{Was wir heute lernen werden}

\begin{itemize}
 \item Schleifen
 \item Iteratoren
 \item Verschiedene Ausführungsmodi
 \item Single vs. Multi Threading
 \item Wie man Pools synchronisiert
 \item Wie man linearen Code in parallelen überführt
\end{itemize}

\end{frame}

{
\usebackgroundtemplate{\pgfuseimage{bgdark}}
\section{Schleifen \& Iteratoren}
}

\begin{frame}[fragile]{Schleifen 0 - Was bisher geschah}
\begin{minted}{C}
const char *data[] = { "Peter Arbeitsloser", ... };

  const int length = sizeof(data) / sizeof(data[0]);
  int index = 0;
kopf:
  if (!(index < length)) {
    goto ende;
  }
  const char *name = data[index];
  printf("%i: %s\n", index, name);
  index += 1;
  goto kopf;
ende:
\end{minted}
\end{frame}

\begin{frame}[fragile]{Schleifen 1 - Was verbessert wurde}
\begin{minted}{C}
const char *data[] = {
    "Peter Arbeitsloser",
    "Sandra Systemadministratorin",
    "Peter Koch",
};

  const int length = sizeof(data) / sizeof(data[0]);

  for (int index = 0; index < length; index++) {
    const char *name = data[index];
    printf("%i: %s\n", index, name);
  }
\end{minted}
\end{frame}

\begin{frame}[fragile]{Schleifen 2}
Die Ausgangslage der folgenden Beispiele:

\begin{minted}{rust}
#[allow(non_upper_case_globals)]
const data: [&str; 3] = [
    "Peter Arbeitsloser",
    "Sandra Systemadministratorin",
    "Peter Koch",
];
\end{minted}
\end{frame}

\begin{frame}[fragile]{Schleifen 3 - While}
\begin{minted}{rust}
    let mut index = 0;
    let length = data.len();
    while index < length {
        println!("{}: {}", index, data[index]);
        index += 1
    }
\end{minted}
\end{frame}

\begin{frame}[fragile]{Schleifen 4 - foreach}
\begin{minted}{rust}
    for name in &data {
        println!("{}", name);
    }
\end{minted}
\end{frame}

\begin{frame}[fragile]{Iteratoren 0}
\begin{minted}{rust}
    let iterator = data.iter();
    iterator.for_each(|name| {
        println!("{}", name);
    });
\end{minted}
\end{frame}

{
\usebackgroundtemplate{\pgfuseimage{bgdark}}
\section{Modes of execution}
}

\begin{frame}[fragile]{Programming is ...}
... about solving problems

Examples:
\begin{itemize}
  \item Copy data
  \item Enhance audio
  \item Distribute messages
  \item Store data
  \item Prepare thumbnails
\end{itemize}

Key is understanding the problem
\end{frame}

\begin{frame}[fragile]{Single thread}
How to do more than one thing at the time?

\begin{itemize}
  \item Linear if tasks are short enough
  \item Polling
  \item Event driven (select/epoll)
  \item Hardware SIMD
\end{itemize}
\end{frame}

\begin{frame}[fragile]{Multi Threading}
Let's add another level of abstraction
\begin{itemize}
  \item spawn / join: handle lists of JoinHandles
  \item pools \begin{itemize}
      \item job queue (the one we look at)
      \item Workstealing (rayon)
      \item futures
    \end{itemize}
\end{itemize}

New problems: synchronization and communication

\end{frame}

{
\usebackgroundtemplate{\pgfuseimage{bgdark}}
\section{Implementation}
}

\begin{frame}[fragile]{Send and Sync}
Rusts "pick three" (safety, speed, concurrency)

\begin{verbatim}
Trait std::marker::Send
\end{verbatim}
Types that can be transferred across thread boundaries.

\begin{verbatim}
Trait std::marker::Sync
\end{verbatim}
Types for which it is safe to share references between threads.

\end{frame}

\begin{frame}[fragile]{Crates}
Let's add another level of abstraction
\begin{itemize}
  \item spawn / join: handle lists of JoinHandles
  \item pools \begin{itemize}
      \item job queue (the one we look at)
      \item Workstealing (rayon)
      \item futures
    \end{itemize}
\end{itemize}

New problems: synchronization and communication

\end{frame}

{
\usebackgroundtemplate{\pgfuseimage{bgdark}}
\section{Examples}
}
\begin{frame}[fragile]{Channel}
\begin{minted}{rust}
use threadpool::ThreadPool; use std::sync::mpsc::channel;

let n_workers = 4; let n_jobs = 8;
let pool = ThreadPool::new(n_workers);

let (tx, rx) = channel();
for _ in 0..n_jobs {
    let tx = tx.clone();
    pool.execute(move || {
        tx.send(1).expect("channel will be there");
    });
}
drop(tx);

assert_eq!(rx.iter().take(n_jobs).fold(0, |a, b| a + b), 8);
\end{minted}
\end{frame}



{
\usebackgroundtemplate{\pgfuseimage{bgdark}}
\section{Code to iterators}
}
\begin{frame}[fragile]{Collect from channnel}
v\_len stores how many elements
\begin{minted}{rust}
  for _ in 0..v_len {
    if let Some(pi) = rx.recv().unwrap() {
      g.pictures.push( pi );
    } else {
      // Abort because of some error in the thread
      return;
    }
  }
\end{minted}
\end{frame}

\begin{frame}[fragile]{Collect from channnel}
\begin{minted}{rust}
  for pi in rx.iter() {
    if let Some(pi) = pi {
      g.pictures.push( pi );
    } else {
      // Abort because of some error in the thread
      return;
    }
  }
\end{minted}
\end{frame}

\begin{frame}[fragile]{Collect from channnel}
\begin{minted}{rust}
  rx.iter().for_each(|pi| {
    if let Some(pi) = pi {
      g.pictures.push( pi );
    } else {
      // Abort because of some error in the thread
      return;
    }
  });
\end{minted}
\end{frame}

\begin{frame}[fragile]{Collect from channnel}
\begin{minted}{rust}
  g.pictures = rx.iter().map(|pi| {
    if let Some(pi) = pi {
      Ok( pi )
    } else {
      // Abort because of some error in the thread
      Err( () )
    }
  }).collect::<Result<Vec<PictureInfo>, ()>>().unwrap();
\end{minted}
\end{frame}

{
\usebackgroundtemplate{\pgfuseimage{bgdark}}
\section{Some pitfalls}
}
\begin{frame}[fragile]{TcpStream with SGE array jobs}
Question: How many connections will each client open
\begin{minted}{rust}
peer_streams = map.values()
    .filter(|s| s.is_some())
    .map(|s| s.unwrap())
    .map(|(addr, data_port)|
        TcpStream::connect(
            SocketAddr::new(addr, data_port)))
    .filter(|s| s.is_ok())
    .map(|s| s.unwrap())
    .collect();
\end{minted}
\end{frame}

{
\usebackgroundtemplate{\pgfuseimage{bgdark}}
\section{Fragen}
}

%{
%\usebackgroundtemplate{\pgfuseimage{bgdark}}
%\section{Workshop time}
%}



%\begin{frame}[fragile]{Warum noch eine Sprache?}
%  \begin{itemize}
%    \item Es ist schwer sicheren und korrekten Code zu schreiben.
%    \item Es ist schwierig parallelen Code zu schreiben.
%  \end{itemize}
%
%\begin{minted}{C}
%char *pi = "3.1415926f32";
%while(1) {
%    printf("wie vielte Stelle? ");  err = scanf("%d", &stelle);
%
%    if (err == 0 || errno != 0) {
%      printf("invalid entry\n");    while (getchar() != '\n');
%      continue;
%    }
%
%    printf("Eingabe: %d\n", stelle);
%    printf("Gewünschte Stelle: '%c'\n", pi[stelle]);
%}
%\end{minted}
%\end{frame}





% ----------------------------------------------------------------- %

{
\setbeamertemplate{footline}{}
\pgfdeclareimage[width=\paperwidth]{bg}{background-inverted.pdf}
\usebackgroundtemplate{\pgfuseimage{bg}}
\begin{frame}[standout]
	\begin{centering}
	{\Huge Danke für eure Aufmerksamkeit!}\\
	{\normalsize Stefan Schindler @dns2utf8 }\\
  {\normalsize Happy hacking! Bitte fragt Fragen! }\\
	{\footnotesize Folien \& Beispiele: \url{https://github.com/dns2utf8/thread-pools-and-iterators}}\\
	%{\footnotesize Examples: \url{https://github.com/coredump-ch/intro-to-rust/tree/master/examples}}\\
	\end{centering}
\end{frame}
}

\end{document}
